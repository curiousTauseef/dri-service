\chapter{Summary and future work}
\label{cha:conclusions}

The main objective of this thesis was creation of a~data reliability
and integrity (DRI) service that monitors the availability and integrity
of data stored on heterogeneous cloud storage resources. As it was shown,
the resulting advantages of moving the data to the cloud suggest that widespread
use of cloud storage seems inevitable. However, this new approach is not free
from dangers. Recent cloud provider failure and malicious corruption reports
\cite{amazon-downtime1, amazon-downtime2, gmail-downtime, docs-downtime} 
show that
one cannot fully entrust the data to the cloud provider. But cloud storage brings
new challenges for ensuring data security. Therefore, computer industry seeks for
innovative tools and methods that could lower the risk associated with the current
trend. The idea oscillates around harnessing multiple cloud storage providers
and replicate the data among them. This approach creates a~new layer of abstraction
in accessing the data -- cloud storage federation. While storing many copies of 
data on different cloud providers significantly reduces the risk of data loss, it
is still needed to detect data problem. Hash-based checksums and error correcting codes
are the industry standard methods of ensuring data integrity. However, cloud storage
introduces obstacles against applying them, because data is stored remotely and cloud
providers charge fees for outbound network transfer. As a~result, for instance, 
validating a~file by computing its SHA-512 hash based on the full content of it can
significantly raise operational costs. Additionally, network latency and throughput
affect the data access.\\

In this work different approaches for ensuring data integrity in cloud storage were
presented. On-going research effort focuses on selectively validating the content of
data and detecting corruption only on some level of probability. Discussed validation
schemes propose different improvements of the outlined approach such as encrypting
the content of data or they assume the existence of the element that performs computation
on data without transfering it to the verifying peer. However, in the scope of VPH-Share
project, the data stored on remote cloud resources cannot be altered, as well as no
computing element exist on cloud provider site.\\

As a~result, in this thesis we aim to address the above mentioned issues with
creation of a~service that is periodically monitoring the availability and integrity
of data and notifies the owner in case of errors. It was successfully designed,
implemented and deployed in production environment of VPH-Share project. However,
at this stage of project, DRI is a~work in progress and not all of its requirements
outlined in VPH-Share deliverable are already met. The core functionality is up and
running, but data replication mechanisms are scheduled for the second phase of the
project. Nevertheless, this thesis objectives have been evaluated and the results
of executing a~test scenario were outlined in the chapter devoted to verification
and testing.\\

The summary of the main objectives of this thesis and proposed solutions how they were
addressed are presented in table \ref{tab:objectives-solutions}.

\begin{table}[h!]
\centering
\begin{tabular}{|l|l|}
	\hline
	Objective & Proposed solution \\ \hline \hline
	Manage data availability and integrity in the cloud & Periodically monitor data integrity status in the cloud \\ \hline
	Manage data recovery in the cloud & Provide data replication across cloud storage providers \\ \hline
	Efficient data validation in the cloud & Probabilistic data validation algorithm\\ \hline
\end{tabular}
\caption{Summary of this thesis objectives and proposed solution how to meet them.}
\label{tab:objectives-solutions}
\end{table}



\section{Future work}
While working on this thesis, we identified some ideas and tasks that are connected
to this work, but were not taken into consideration. However, they could be worth
continuation, so we outline them here. We divided them in two groups. The first presents
possibilities of enhancements and improvements to this work:

\begin{enumerate}
\item Design and implementation of automatic data replication module. The idea
is to take advantage of and combine both data replication and validation. As soon as
DRI discovers integrity errors, it will recover from them automatically by restoring
the data from other, non-affected replicas. During data recovery, corrupted replica
should be excluded from set of replicas available to VPH users.

\item Investigation and implementation of possible improvements to the validation
algorithm. At the time of writing this thesis, cloud storage interfaces are of
limited flexibility. Every noncontiguous block of data has to be requested with 
separate HTTP request. It can significantly affect the efficiency and throughput
of validation algorithm. In the current implementation, DRI asynchronously sends
a~set of HTTP requests to reduce network latency. If cloud providers introduced
support for requesting multiple noncontiguous blocks of data in single HTTP call
or for emerging SPDY web protocol, it would be beneficial to add support for it
in DRI. Another idea it to perform data validation against multiple cloud providers
simultaneously. Separate blocks of data should be reqested from different replicas.
Other implementation improvements could also be investigated.

\item Design and investigation how to combine DRI with LOB federated data access
(LOBCDER) into one component. LOBCDER provides federated data access for the VPH-Share
Cloud Platform -- all the requested data flows through this component. Many limitations
to DRI design came out from separating these two components by design. In case of
ensuring data integrity the combined component could perform data validation
on the fly as the data is requested and retrieved from cloud storage. When data corruption
occur, it could automatically recover by restoring the data from the remaining replicas.
Moreover, it could perform data encryption while storing in the cloud. As a~result,
validation algorithm would not be constrained with the requirement to store the data
in original form, as well as it could be substituted with proofs or retrievability
scheme.
\end{enumerate}
 
The other group addresses the problem of ensuring data integrity in cloud storage
in general and abstracts away from VPH-Share project context. The concept monitoring
data integrity in DRI service has a~potential for being applied as a~part of many
software solutions. The ideas in this group are the following:

\begin{enumerate}
\item Extract the implementation of data validation mechanism and abstract it away
from the context of VPH-Share project. It would be beneficial to take out DRI
functionality and share it as open source solution. Needless to say that its architecture
should be redesigned and implementation refactored. The design should clearly specify
its dependencies and provide a~couple of implementations out of the box. Currently,
DRI has two core dependencies -- metadata registry and notification service. Metadata
registry stores all the metadata related to data validation and as such could be
implemented as file, database (relational or no-sql) or stored itself on cloud storage
in encrypted form. Notification service informs the user about discovered data integrity
problems and could be implemented in form of email sender, XMPP protocol bot or sms
gateway. Validation algorithm should also be pluggable as different use cases have 
different requirements and limitations.

\item Explore new ways of ensuring data integrity in cloud storage or design a~new
validation algorithm that would satisfy the requirements outlined in this thesis. It is
an on-going research in this field as no standards ways of monitoring data integrity in
the cloud emerged.
\end{enumerate}

The above future work suggestions provide only a~brief description and probably does
not exhaust the subject. However, they are provided as an~inspiration for the broad
spectrum of improvement possibilities.
